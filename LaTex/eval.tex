To adequately test our infrastructure, we set out to exploit the specific methods that the web application firewall was built to defend against. This is done through the login page of our frontend environment. A successful evaluation will be represented by any attack execution being redirected via a HTTP 403 response, indicating that the attack was successfully detected and blocked by our web application firewall.

The first exploit we chose was an SQL injection, specifically a tautology attack. This attack aims to break the conditional statement for a typical user name/password login combination on an SQL backend server such that when the query is made through the login page to check if the user exists the condition will always be true, thereby allowing the attacker to log in as any user that exists~\cite{tautology}.

The second exploit we chose was a cross-site scripting (XSS) attack. This attack attempts to inject malicious scripts into input sections or source code of a website~\cite{xss}. This attack will also be tested by appending an exploit to the end of our URL. If the threat actor is able to successfully execute their attack, they would be able to run arbitrary scripts in the context of the user's browser. This could lead to a range of malicious activities, such as stealing cookies, session tokens, or other sensitive information, defacing the website, or redirecting users to malicious sites. Our testing will involve various payloads to simulate real-world XSS attack vectors and assess the vulnerability of our web application to these types of exploits.

The final exploit we will be testing is command injection. This exploit attempts to bypass a input field in such a way that allows the attacker to execute arbitrary operating system commands on the server ~\cite{cmd_injection}. If an attacker is able to successfully perform this attack, they can compromise the web server, exfiltrate data, and establish persistence by injecting a backdoor into our system ~\cite{backdoor}. 