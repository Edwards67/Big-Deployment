The 'Weather and Jokes' web application employs Kubernetes and NGINX within its architecture to provide defenses against several cybersecurity threats. 
We are specifically defending against unauthorized access, SQL injections, Cross-Site Scripting (XSS), session hijacking, and Denial of Service (DoS) attacks. 
In contemplating the perceived capabilities of potential attackers, we recognize that they are likely to possess a high degree of technical sophistication, equipped with the skills necessary to exploit vulnerabilities typical of web applications. 
These attackers might use reconnaissance tools to gain information, and advanced techniques for SQL injections, craft XSS attacks, and leverage automated tools to execute DoS attacks.

To counter these threats, our application utilizes a load balancer within the Kubernetes environment to effectively manage traffic volumes that could potentially lead to DoS attacks. 
This load balancer helps distribute incoming traffic evenly across available servers, preventing any single server from becoming overwhelmed
Additionally, NGINX, configured as a reverse proxy, is important in safeguarding against unauthorized access and filtering out malicious requests. It acts as a gatekeeper, routing all incoming traffic through its server and using ModSecurity-a web application firewall(Talked more about later)—to identify and block threats identified from patterns typical in SQL injections and XSS. 
Moreover, NGINX enhances the security of user sessions by managing encrypted connections, a crucial feature for preventing session hijacking. 
These encrypted channels ensure that session tokens, critical for maintaining user session integrity, are not intercepted or tampered with by unauthorized parties.
Through the strategic implementation of Kubernetes and NGINX, our application is well-prepared to defend against a wide spectrum of cybersecurity threats.
