  The first test run was an SQL Injection on the Login page, using the expression " admin' OR '1'='1' -- " in the user input with " password"  in the password input. Normal behavior would just redirect the page back to itself, however because it detected an SQL Injection the page was forwarded to a 403 Forbidden page, and the suspicious activity was logged, with  the time of attempted access, the attempted username, the attempted password, and the IP address of the attacker. The next attack we used was an SQL injection in the password input where user input was " user " , and password input was " wrongpassword' OR 'a'='a " again we got the same results as the first attempt, a it forwarded the client to a 403 forbidden page was and the suspicious activity was logged. 

  The second test we ran was cross site scripting or XSS, so in normal behavior the pages would redirect normally, so 10.102.68.91/about to 10.102.68.91/home, however in testing we added an XSS script at the end of the url to test the behavior of our webpages. The command we used was http://10.102.67.91/about=</script><script>alert(0)</script>, an XSS script that would thrown an alert on the webpage. However our defense worked as intended redirecting to a 403 forbidden page and did not create the XSS alert as the attacker intended. The suspicious activity was logged and given to our security team so they could block the IP address of the potential attacker. This 403 error actually gets thrown any time there is a reference using </script>. The normal activity for the webpage is to redirect to a url not found, for example if you enter 10.102.68.91/about=user it will redirect to URL not found, so any XSS or added script will be blocked by our firewall. We made sure to test each of our pages, home, about, manual, and login, and each page passed as intended.

    The third test we ran was for command injection we again used on our login page. In normal behavior a successful login attempt would redirect back to the login page. The attack we tested command injection is a type of attack that takes advantage of a program using a operating system call, for example if our program ran ping and displayed results of the ping, an attacker would be able to input ' 8.8.8.8;ls ' to display information about whats inside their current directory. The ' ;ls ' is the attack and 8.8.8.8 would be the normal input. So in our test we used the username input ' user;ls ' and password as ' password ', the firewall successully blocked this attack and redirected the page to a 403 forbidden page and logged the suspicious activity, alerting our monitoring team and indicating another successful test. 
