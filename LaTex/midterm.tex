\documentclass[letterpaper,twocolumn,10pt]{article}
\usepackage{usenix-2020-09}

% to be able to draw some self-contained figs
\usepackage{tikz}
\usepackage{amsmath}

% inlined bib file
\usepackage{filecontents}

%-------------------------------------------------------------------------------
\begin{document}
%-------------------------------------------------------------------------------

%don't want date printed
\date{}

% make title bold and 14 pt font (Latex default is non-bold, 16 pt)
\title{\Large \bf Implementing a Secure Microservice Architecture:\\
  NGINX \& Kubernetes Deployment}

%for single author (just remove % characters)
\author{
{\rm Orchestration: Alexander Moomaw}\\
Eastern Washington University
\and
{\rm Services: Andre Ramirez}\\
Eastern Washington University
\and
{\rm Logging: Brendan Hopkins}\\
Eastern Washington University
\and
{\rm Services: Beighlor Martinez}\\
Eastern Washington University
\and
{\rm Defense: Cameron Olivier}\\
Eastern Washington University
\and
{\rm Frontend: Dillon Pikulik}\\
Eastern Washington University
\and
{\rm Project Management: Chelsea Edwards}\\
Eastern Washington University
}
% copy the following lines to add more authors
% \and
% {\rm Name}\\
%Name Institution
 % end author

\maketitle
%-------------------------------------------------------------------------------
\begin{abstract}
%-------------------------------------------------------------------------------
This paper presents a secure and scalable deployment architecture for web server environments using Kubernetes. 
The deployment incorporates a proxy and web application firewall for enhanced security and modular design. 
Additionally, we introduce our solution to logging and IP banning mechanisms to monitor and restrict unauthorized access. 
Our approach ensures high availability, load balancing, and robust defense against threats, offering a comprehensive framework for modern web applications.
\end{abstract}

%-------------------------------------------------------------------------------
\section{Introduction}
%-------------------------------------------------------------------------------
Ensuring the security and scalability of web server environments is a challenge in modern digital infrastructure. 
As cyber threats become more advanced, it's crucial to implement solutions that protect against unauthorized access 
while also maintaining high availability and performance. 

To address these challenges, we propose a comprehensive deployment architecture by first utilizing Kubernetes, an open-source platform designed for automating deployment, scaling, 
and management of containerized applications ~\cite{kubernetes}. With this platform, we ensure high-availability to prospective users and a modular
environment to build and scale upon as we please. 

To enhance our microservice architecture, we introduced an additional abstraction layer using an NGINX proxy hosted locally on the server. 
This proxy server fetches content from our Kubernetes load balancer, serving as a gateway between our internal services and external communications. 
To mitigate potential threats, we integrated ModSecurity, a well-known web application firewall (WAF), into our proxy ~\cite{modsecurity}. 
ModSecurity monitors and blocks common exploitation methods by referencing a comprehensive rule list that addresses the OWASP Top Ten vulnerabilities ~\cite{owasp_top_ten}.

Exploiting our endpoint services isn't the only attack vector within our environment. 
To address this, we implemented host-level logging to monitor network traffic and identify IP addresses not communicating via SSH or HTTP. 
Additionally, we applied an automated process to block malicious IP addresses, enhancing our overall security posture.



%-------------------------------------------------------------------------------
\section{Orchestration}
%-------------------------------------------------------------------------------
\input{orchestration}

%-------------------------------------------------------------------------------
\section{Floating Figures and Lists}


Our Website is called ‘Weather and Jokes’, and includes four pages: Home, Contact, FTP Manual, and Login. The Home page runs two services, ‘Get a Joke’ and ‘Weather Checking’. The Contact page lists each of our team members and their roles within the group. The FTP Manual page gives instructions to our team members on how to use FTP. The FTP Manual page is meant to be a distraction for threat actors and is not fundamental to the actual Website, but rather a ploy to try and catch attackers attempting to use FTP. The final page, Login, takes in user input and redirects back to the log-in page. The log-in page never actually connects to any backend service, this page is used to try and catch potential threat actors attempting to use attacks such as SQL Injections.

%-------------------------------------------------------------------------------
%-------------------------------------------------------------------------------
\section{Logging and IP Gathering}
To enhance the security of our web server environment, we implemented a robust mechanism 
for gathering and logging IP addresses. This approach involves capturing IP addresses 
attempting to access our server on ports other than 22 (SSH) and 80 (HTTP) and maintaining 
comprehensive logs for further analysis. 
The IP gathering mechanism is designed to identify unauthorized access attempts by capturing IP
addresses that try to connect to non-standard ports. This is achieved through a script that utilizes
tcpdump to monitor network traffic and filter out unwanted connections. The captured IP addresses 
are stored in a cumulative list, which can be used to block malicious IPs via IP tables.
In addition to gathering IP addresses, we implemented an hourly logging mechanism to maintain detailed
records of network activity. This involves running a script that logs all IP traffic, excluding ports 22 and 80,
and saves the logs with timestamps. These logs provide valuable data for analyzing traffic patterns and identifying 
potential security threats.
To ensure the effectiveness of our IP gathering and logging mechanisms, the captured IP addresses and logs are periodically 
reviewed. This verification process involves accessing the log files to confirm that the security measures are functioning as 
intended and to identify any anomalies in network traffic.
The 'Weather and Jokes' web application employs Kubernetes and NGINX within its architecture to provide defenses against several cybersecurity threats. 
We are specifically defending against unauthorized access, SQL injections, Cross-Site Scripting (XSS), session hijacking, and Denial of Service (DoS) attacks. 
These measures are crucial for maintaining both the integrity and availability of our services and for ensuring that user data remains secure against potential exploitation.
In contemplating the perceived capabilities of potential attackers, we recognize that they are likely to possess a high degree of technical sophistication, equipped with the skills necessary to exploit vulnerabilities typical of web applications. 
These attackers might use advanced techniques for SQL injections, craft XSS attacks, and leverage automated tools to execute DoS attacks.

To counter these threats, our application utilizes a load balancer within the Kubernetes environment to effectively manage traffic volumes that could potentially lead to DoS attacks. 
This load balancer helps distribute incoming traffic evenly across available servers, preventing any single server from becoming overwhelmed
Additionally, NGINX, configured as a reverse proxy, is pivotal in safeguarding against unauthorized access and filtering out malicious requests. It acts as a gatekeeper, routing all incoming traffic through its server and using ModSecurity-a web application firewall(Talked more about later)—to identify and block threats identified from patterns typical in SQL injections and XSS. 
Moreover, NGINX enhances the security of user sessions by managing encrypted connections, a crucial feature for preventing session hijacking. 
These encrypted channels ensure that session tokens, critical for maintaining user session integrity, are not intercepted or tampered with by unauthorized parties.
Through the strategic implementation of Kubernetes and NGINX, the 'Weather and Jokes' application is well-prepared to defend against a spectrum of cybersecurity threats.

%-------------------------------------------------------------------------------

%-------------------------------------------------------------------------------
\section{Floating Figures and Lists}

To adequately test our infrastructure, we set out to exploit the specific methods that the web application firewall was built to defend against. This is done through the login page of our frontend environment. A successful evaluation will be represented by any attack execution being redirected via a HTTP 403 response, indicating that the attack was successfully detected and blocked by our web application firewall.

The first exploit we chose was an SQL injection, specifically a tautology attack. This attack aims to break the conditional statement for a typical user name/password login combination on an SQL backend server such that when the query is made through the login page to check if the user exists the condition will always be true, thereby allowing the attacker to log in as any user that exists~\cite{tautology}.

The second exploit we chose was a cross-site scripting (XSS) attack. This attack attempts to inject malicious scripts into input sections or source code of a website~\cite{xss}. This attack will also be tested by appending an exploit to the end of our URL. If the threat actor is able to successfully execute their attack, they would be able to run arbitrary scripts in the context of the user's browser. This could lead to a range of malicious activities, such as stealing cookies, session tokens, or other sensitive information, defacing the website, or redirecting users to malicious sites. Our testing will involve various payloads to simulate real-world XSS attack vectors and assess the vulnerability of our web application to these types of exploits.

The final exploit we will be testing is command injection. This exploit attempts to bypass a input field in such a way that allows the attacker to execute arbitrary operating system commands on the server ~\cite{cmd_injection}. If an attacker is able to successfully perform this attack, they can compromise the web server, exfiltrate data, and establish persistence by injecting a backdoor into our system ~\cite{backdoor}. 
%-------------------------------------------------------------------------------
\bibliographystyle{plain}
\bibliography{refs}

%%%%%%%%%%%%%%%%%%%%%%%%%%%%%%%%%%%%%%%%%%%%%%%%%%%%%%%%%%%%%%%%%%%%%%%%%%%%%%%%
\end{document}
%%%%%%%%%%%%%%%%%%%%%%%%%%%%%%%%%%%%%%%%%%%%%%%%%%%%%%%%%%%%%%%%%%%%%%%%%%%%%%%%

%%  LocalWords:  endnotes includegraphics fread ptr nobj noindent
%%  LocalWords:  pdflatex acks
