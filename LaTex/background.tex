This is a sample document, you next section should probably be 
``background" or ``motivating example".

Footnotes should be places after punctuation characters, without any
spaces between said characters and footnotes, like so.%
\footnote{Remember that USENIX format stopped using endnotes and is
  now using regular footnotes.} And some embedded literal code may
look as follows.

\begin{verbatim}
int main(int argc, char *argv[]) 
{
    return 0;
}
\end{verbatim}

Now we're going to cite somebody. Watch for the cite tag. Here it
comes. Arpachi-Dusseau and Arpachi-Dusseau co-authored an excellent OS
book, which is also really funny~\cite{arpachiDusseau18:osbook}, and
Waldspurger got into the SIGOPS hall-of-fame due to his seminal paper
about resource management in the ESX hypervisor~\cite{waldspurger02}.

The tilde character (\~{}) in the tex source means a non-breaking
space. This way, your reference will always be attached to the word
that preceded it, instead of going to the next line.

And the 'cite' package sorts your citations by their numerical order
of the corresponding references at the end of the paper, ridding you
from the need to notice that, e.g, ``Waldspurger'' appears after
``Arpachi-Dusseau'' when sorting references
alphabetically~\cite{waldspurger02,arpachiDusseau18:osbook}. 

It'd be nice and thoughtful of you to include a suitable link in each
and every bibtex entry that you use in your submission, to allow
reviewers (and other readers) to easily get to the cited work, as is
done in all entries found in the References section of this document.

Now we're going take a look at Section~\ref{sec:figs}, but not before
observing that refs to sections and citations and such are colored and
clickable in the PDF because of the packages we've included.
