We implemented our website using a Dockerfile pulling an image of Ubuntu which installs Flask that we use as our webserver. You can find more information on backend code and the Flask Server in the services section. We decided to use Flask based on previous experience with this service and Docker so we could containarize our project for added defense. By adding Docker we provided isolation from a singular host and allowed us to add defensive features without effecting our entire system. Our Website is called ‘Weather and Jokes’, and includes four pages: Home, Contact, FTP Manual, and Login. The Home page runs two API services, ‘Get a Joke’ and ‘Weather Checking’, the Contact page lists each of our team members and their roles within the group, the FTP Manual page gives instructions to our team members on how to use FTP and the final page, Login, takes in user input and redirects back to the login page. 

The FTP Manual page is meant to be a distraction for threat actors and is not fundamental to the actual Website, but rather a ploy to try and catch attackers attempting to use FTP. The login page never actually connects to any backend service and is also used to try and catch potential threat actors attempting to use attacks such as SQL Injections. These features allow our webpage to act more as a honeypot gathering information on potential attackers, and allowing our defense team to monitor/log their activities and if necessary block their IP address. 
