We have established a comprehensive security protocol to monitor and block unauthorized access attempts to our web server. This involves capturing IP addresses attempting to access our server on ports other than 22 (SSH) and 80 (HTTP) and maintaining comprehensive logs for further analysis.

The process begins by clearing the log file \texttt{/var/log/honeypot/block_ip.log} to ensure new entries are recorded clearly. The script then checks for the existence of a cumulative IP list file at \texttt{/var/log/honeypot/ip_list_all.txt}, which contains IP addresses flagged for suspicious activity. If this list exists, the script reads each IP address and adds a \texttt{DROP} rule to \texttt{iptables}, effectively blocking incoming traffic from these addresses. The success or failure of each attempt to block an IP address is logged with a timestamp in the \texttt{block_ip.log} file. If the cumulative IP list is not found, an error message is logged and the script exits.

By employing this method, we aim to enhance our ability to detect and respond to potential security threats effectively. The combination of real-time IP address gathering, systematic logging, and automated blocking helps ensure that our web server environment remains secure and resilient against unauthorized access attempts.
