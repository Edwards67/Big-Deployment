We have 2 services, both run in \verb+flask_example.py+ file and are presented on the \verb+index.html+ page. \verb+Flask+ is the resource we used to host the website locally. \verb+Flask+ is part of \verb+Python+, which is easy to work with and implement, that was the main reason for choosing \verb+Flask+.
One of our services is utilizing a weather \verb+API+. This \verb+API+ has the weather in different geo locations around the world. We use this \verb+API+ to find the weather of a website’s geo location. The user has a \verb+textbox+ on the \verb+index.html+ page that they can enter the websites \verb+URL+. After they enter the \verb+URL+ they can hit the \verb+button+ to get the weather. We ask for user input of a website's \verb+URL+ and output the current weather of that website. This is done by taking the \verb+URL+ and running \verb+gethostbyname+ to get the IP address of that \verb+URL+. Now after we get the IP address, we run a \verb+whois+ command on the IP address. This command will give us a lot of information on the IP address, but we parse out the physical address associated with the IP. After the physical address is parsed and formatted for the \verb+API+ we call \verb+(https://geocoding.geo.census.gov/) API+ and get returned the \verb+X,Y+ geo coordinates of that address ~\cite{GeoLocationAPI}. Now that we have geo coordinates we can send those to the weather \verb+API+. The weather \verb+API+ is at \verb+https://api.weather.gov/points/+, where we input the coordinates at the end of the \verb+URL+ to get back the weather ~\cite{WeahterAPI}. After we send the coordinates, we receive the weather of that location, which is the location of the initially given website. All of these command are automated in the \verb+flask_example.py+ file. I chose to implement this service because I have done it in the past and had very well-documented files that aided in quick implementation. This in return aided the rest of the team by speeding up the process of deployment. This service also is very large and can be used for about every website, this could keep the attacks busy, possibly giving information away about themselves.

Another \verb+API+ service that was utilized was the\verb+ /joke+ route in the \verb+Flask+ application serving a random dad joke to the user. When accessed, the function \verb+joke()+ is executed, which sets the \verb+API URL https://icanhazdadjoke.com/+ and defines a \verb+cURL+ command to request a joke in plain text format ~\cite{DadJokesAPI}. The \verb+subprocess.run+ function runs this command and captures the output, which contains the joke. Finally,  \verb+result.stdout+ is returned, sending the joke back to the user's browser. Additionally, there is a clickable button in \verb+index.html+ that allows users to retrieve a joke, providing a simple and interactive way to display a random dad joke. The reason this service was chosen was to have clickable and interactable features on the websites so that targets on the website are encouraged to click more and give potential information.

